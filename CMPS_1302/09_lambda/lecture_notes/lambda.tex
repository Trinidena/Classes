\documentclass{beamer}

\usepackage[english]{babel}
\usepackage{minted}
\newcommand{\mil}[1]{\mintinline{java}{#1}}
\usepackage{upquote}
\usepackage{graphicx}
\usepackage{tikz}
\usepackage{hyperref}
\usetikzlibrary{matrix,backgrounds}
\usepackage{hyperref}

\title[Lambda Expressions and Functional Programming]{Java}
\subtitle{Lambda Expressions and Functional Programming} % (optional)

\author[]{Dalton State College} 

%\institute[Dalton State College]

\date[T. Gonzalez]{T. Gonzalez}

% If you wish to uncover everything in a step-wise fashion, uncomment
% the following command: 

%\beamerdefaultoverlayspecification{<+->}


\begin{document}

\begin{frame}

	\titlepage
	
\end{frame}

\begin{frame}

    \frametitle{Lambda Expressions}
    
    A lambda expression is an anonymous method, that is, a method without a name.
    
\end{frame}

\begin{frame}[fragile]
    
    \frametitle{Lambda Expression Syntax}
    
        A lambda expression consists of:
        
        \begin{enumerate}
        
            \item A comma-separated list of formal parameters enclosed in parentheses.
                \begin{itemize}
                    \item The types of the parameters can be omitted.
                    \item The parentheses can be omitted if there is only one parameter.
                \end{itemize}
            \bigskip    
            \item The arrrow token. \mil{->}
            \bigskip            
            \item A body, which consists of a single expression or a statement block.
                \begin{itemize}
                    \item If the body consists a single expression, the expression is evaluated and the result is returned.
                    \item If the body uses statements, then the statements must be enclosed in curly braces.
                    \item Method calls do not have to be enclosed in braces.
                \end{itemize}
            
        \end{enumerate} 
\end{frame}

\begin{frame}[fragile]
    
    \frametitle{Lambda Expression Examples}
    
    \mil{n -> 3 * n + 2}
    
    \bigskip
    
    \mil{(x, y) -> x > y}

    \bigskip
        
    \mil{z -> System.out.println(z)}
    
    \bigskip

\begin{minted}[fontsize=\footnotesize]{java}
x -> {if(1 < x && x < 5)
          return true;
      else if(x == 2)
          return false;
      else
          return true;}
\end{minted}

\end{frame}

\begin{frame}

    \frametitle{Using Lambda Expressions to Process Lists and Maps}
    
    Lists such as an \mil{ArrayList}s can be used processed using lambda expressions.
    
    \bigskip
    
    Lambda expressions can be used in conjunction with the following methods:
    
    \bigskip
    
        \begin{itemize}
        
            \item \mil{replaceAll()}
            
            \item \mil{removeIf()}
            
            \item \mil{forEach()}
        
        \end{itemize}
     
\end{frame}

\begin{frame}

    \frametitle{The \mil{replaceAll()} Method}
    
    Given a \mil{List} named \mil{list} and a lambda expression \mil{lambda} the statement \mil{list.replaceAll(lambda);} calls the lambda expression once for every item in the list and then stores the result back in the list.
    
    \bigskip
    
    The value returned by the lambda expression must be the same type as the type of the list.
    
    \bigskip
    
    See ReplaceAllDemo.java.
       
\end{frame}

\begin{frame}

    \frametitle{The \mil{removeIf()} Method}
    
    Given a \mil{List} named \mil{list} and a lambda expression \mil{lambda} the statement \mil{list.removeIf(lambda);} calls the lambda expression once for every item in the list and then removes the item when the result of the lambda expression is \mil{true}.
    
    \bigskip
    
    See RemoveIfDemo.java.
       
\end{frame}

\begin{frame}

    \frametitle{Method References}

    Method references are compact, easy-to-read lambda expressions for methods that already have a name.
    
    \bigskip
    
    Lambda expression:  \mil{x -> System.out.println(x)}
    
    \bigskip
    
    Method reference:  \mil{System.out::println}
    
    \bigskip
    
    There are four types of method references:
    
    \bigskip
    
\scriptsize{    
\begin{tabular}{p{1.5in}l}
Reference to a static method &	ContainingClass::staticMethodName\\
 & \\
Reference to an instance method of a particular object &	containingObject::instanceMethodName\\
 & \\
Reference to an instance method of an arbitrary object of a particular type	& ContainingType::methodName\\
 & \\
Reference to a constructor	& ClassName::new 
\end{tabular}   }
\end{frame}

\begin{frame}

    \frametitle{The \mil{forEach()} Method}
    
    Given a \mil{List} named \mil{list} and a lambda expression \mil{lambda} the statement \mil{list.forEach(lambda);} calls the lambda expression once for every item in the list.
    
    \bigskip
    
    See ForEachDemo.java.
       
\end{frame}

\begin{frame}

    \frametitle{In-Class Problem}
    
    Write a program that declares and initializes an \mil{List} of \mil{Integer}s with at least five elements.  Use a lambda expression to replace all of the elements in the list with the length of the list.
    
\end{frame}

\begin{frame}

    \frametitle{In-Class Problem}
    
    Write a program that declares and initializes a \mil{List} of \mil{Integer}s.  Use a lambda expression to remove all odd multiples of three from the list.
    
\end{frame}

\begin{frame}

    \frametitle{In-Class Problem}
    
    Write a program that declares and initializes a \mil{List} of \mil{OrderedPair} objects.  Use a lambda expression to call the \mil{updateY()} method in each of the objects in the list.
    
\end{frame}

\begin{frame}[fragile]
    
    \frametitle{In-Class Problem}
    
    Write a program that reads the data from the file name\_company\_data.csv into a \mil{HashMap}.  Use lambda expressions to do the following in order:
    
    \begin{itemize}
    
        \item Remove all information of people who have a lowercase a or e in their name.
        \item Capitalize the following letters in company names: i, o, u, and y.
        \item Print the resulting map.
  
    \end{itemize}
    
\end{frame}

\end{document}